\documentclass[12pt]{amsart} 
\usepackage{geometry}                		
\geometry{letterpaper}                   	
\usepackage{graphicx}				
\newtheorem{thm}{Theorem}
\newtheorem{cor}{Corollary}
\usepackage{amssymb}
\usepackage{dsfont}

\title{Erdos-Sierpinski Problem}
\author{Thomas Lockwood}
%\date{}							


\begin{document}
\maketitle
\thispagestyle{empty}
\begin{abstract}
The Sum of Divisors Function of an integer $n$, denoted $\sigma(n)$ is the sum of positive divisors $n$. The Erdos-Seirpenski Problem asks: Are there infinitely many solutions to $\sigma(n) = \sigma(n+1)$? Erdos claimed that there were, however it has yet to be proven. Spira defined the Sum of Divisors Function for the Gaussian Integers, denoted $g\sigma(n)$. Combining the two, we look into the problem: Are there infinitely many solutions to $g\sigma(n) = g\sigma(n+1)$?
\end{abstract}

\section{The Integers}
The Sum of Divisors Function, or sigma function $\sigma(n)$, is the sum of all the positive divisors of an integer $n$. If n has the prime factorization $n = {p_1}^{m_1}{p_2}^{m_2}\cdots{p_N}^{m_N}$ $$\sigma(n) = \sum_{d|{n}}d = \left(\frac{{p_1}^{m_1+1} - 1} {{p_1} - 1}\right) \left(\frac{{p_2}^{m_2+1} - 1}{{p_2} - 1}\right)\cdots \left(\frac{{p_N}^{m_N+1} - 1}{{p_N} - 1}\right) $$ 
The Erdos-Sierpinski Problem is to find solutions to the equations $\sigma(n) = \sigma(n+1)$. The first few solutions are:
$$\sigma(206) = \sigma(207) = 312$$
$$\sigma(957) = \sigma(958) = 1440$$
$$\sigma(1334) = \sigma(1335) = 2160$$
$$\sigma(1364) = \sigma(1365) = 2688$$
$$\sigma(1634) = \sigma(1635) = 2688$$
$$\sigma(2685) = \sigma(2686) = 4320$$
$$\sigma(2974) = \sigma(2975) = 4464$$
$$\sigma(4364) = \sigma(4365) = 7644$$

Guy and Shanks noted that some solutions have the form $$n = 2p,\; \; n + 1 = 3^{m}q$$ when $q$ and $p$ are both prime with $$q = 3^{m+1} - 4, \; \; p=\frac{3^{m}q-1} {2}$$ Yields the solution $n = 14, 206, 19358$ for $m=1,2$, and $4$
Similarly, $$n = 3^{m}q, \; \; n+1 = 2p$$ when $q$ and $p$ are both prime for $$q = 3^{m+1} - 10, \; \; p = \frac{3^{m}q + 1} {2}$$  yield the solutions when $m = 4$ and $5$. However this does not solve the problem since it only produces the 3 and 2 addtional solutions respectively.

\section{The Sigma Fucntion in the Gaussian Integers}
	The ring $\mathds{Z}[\sqrt{-1}]$ is usually called the Gaussian Integers. In Spira's Paper, Spira defined a mutiplicative sum-of-divisors function on the Gaussian Integers which we will denote $g\sigma(z)$. As there are four units in the Gaussian Integers, each prime $\pi$ Gaussian Sigma has 4 associates: $\pi$, $i\pi$, $-\pi$, and $-i\pi$. Therefore in the prime factorization of a Gaussian Integer, $z = \epsilon\pi^{m_1}\cdots\pi^{m_k}$, we choose all the primes to be in the first quadrant, and not on the imaginary axis by factoring out various units. (Then $\epsilon$ is the product of those units). Assuming the $\pi$, are all in the 1st quandrant and not on the Imaginary Axis, Spria defined:$$g\sigma(z)=g\sigma(\pi^{m_1}\cdots\pi^{m_k}) = \prod_{j=1}^{k}\left(\frac{\pi_j^{m_j-1}} {\pi_j - 1}\right) = \prod_{j=1}^{k} \left(1 + \pi_j + \cdots + \pi_j^{m_j}\right)$$ 
Which is evidently multpilicative. 

For example, in $\mathds{Z}[\sqrt{-1}]$, $5=(2+i)(2-i)$. Since $2-i$ is not in the first quadrant, we factor out $-i$ $$5=(2+i)(2-i) = (2+i)(-i)(1+2i) = -i(2+i)(1+2i)$$ 
Then both the primes $2+i$ and $1+2i$ are in quadrant one, and $$g\sigma(5) = (1+(2+i))(1+(1+2i))
= (3+i)(2+2i) = 4+8i$$

We now seek the solutions to the equations: $$g\sigma(z) = g\sigma(z+1)$$ $$g\sigma(z) = g\sigma(z+i)$$ $$g\sigma(z+1+i)$$
We call this the "Erdos-Sierpinski Problem in the Gaussian Integers".

\section{Erdos-Sierpinski Problem in $\mathds{Z}[\sqrt{-1}]$}

Let $n$ be prime, and $z+1$ be the product of two distinct primes $p$ and $q$, then $z = (z + 1) - 1 = pq - 1$. If $p$ and $q$ are of odd parity, then $n = (1+i)$. This would force $pq = 2+i$, a contradiction since $2+i$ is prime and can't be the product of two primes. So either $p$ or $q$ must be of even parity, namely $(1+i)$, WLOG let $p = 1 + i$ and $q=a + bi$, then $z = (1+i)q - 1$. Assume $g\sigma(z) = z+1 = g\sigma(z+1) = \gsigma(1+i) = (2+i)(a + bi +1)$, then we have the equations: $$1 = 2a + 2$$ and $$i(1) = i(a + 2b + 1)$$ By Spira's definition of Complex Divisor Function we are unable to have negative numbers in the domain, therefore leads to a contradiction, so there exists no solutions in this form.

This time let $z+1$ be prime, and $z$ be the product of two distinct primes $p$ and $q$, then $z = (z+1) -1 = pq$. For $z+1$ to be of even parity, then $z = i$, and $i$ is not the product of two primes. That leaves $z$ to be of even parity, so either $p$ or $q$ is $1+i$. WLOG let $p=1+i$, $q=a+bi$, and $z = c + di$, then $(1+i)(a+bi) = (c + di+1) - 1 = c + di$. Assume $g\sigma((1+i)(a+bi)) = (2+i)(a+1+bi) = g\sigma(z+1) = n + 2 = c + di + 1$, then we have the four equations:
$$a -b = c$$ $$i(a + b) = i(d)$$ $$2a + 2 -b = c + 1$$ $$i(a + 1 + 2b) = i(d)$$ Taking $d = a + 1 + 2b = a + b$ we get $b = -1$, however the input may not be negative leading to our contradiction. Thus there are no consecutive integers in this form or any form.
\\
\section{Solutions to $g\sigma(pq) = g\sigma(rs)$}
Furthermore, using Guy and Shanks method with Gaussian Integers we have $z$ and $z+1$ the form of two distinct primes. For the distinct primes $q$ and $p$ (either not equal to $1+i$ or $2+i$), let $z = (1+i)p$ and $z+1 = (2+i)q$. If we let $p = a+bi$ and $q=c+di$, then $z = (z + 1) - 1 = (1+i)(a+bi) = (2+i)(c+di) - 1$.  Assume $g\sigma(z) = g\simga(z+1)$, then $g\sigma(1+i)g\sigma(p) = (2+i)(a + bi + 1) = g\sigma(2+i)g\sigma(q) = (3+i)(c+di+1)$. Splitting the real and imaginary parts we have the four equations: $$ a -b = 2c - d - 1$$ $$(a + b)i = i(c + 2d)$$ $$2a + 2 -b = 3c + 3 - d$$ $$i(a +2b + 1) = i(c + 3d + 1)$$
From the four equations we have: $a = 5$, $b=2$, $c = 3$, and $d = 2$. Since $a+bi$ is prime ($5^2 + 2^2 = 29$), and same holds for for $2+3$.
\\

Suppose we know $a,b,e,f$, then we can solve for $c,d,g,h$.
Let $$n = (a + bi)(c+di), \; \; \; n+1 = (e + fi)(g + hi)$$ and by $g\sigma$ of primes  $$ g\sigma(z) = (a + bi + 1)(c + di + 1) = (ac + a + c -bd) + i(ad + d + bc + b)$$ $$ g\sigma(z+1) = (e + fi + 1)(g + hi + 1) = (eg + e + g - fh) + i(eh + fg + h + f)$$
and we have the four equations 
$$ac -bd = eg -fh - 1$$ $$bc + ad = fg + eh$$ 
$$ac + a + c -bd = eg + e + g - fh$$ 
$$ad + d + bc + b = eh + fg + h + f$$
Thus our solutions: 
$$c = \frac{ea^2 - 2e^2a -ea -a + eb^2 -bf -2ebf + ef^2 + f^2 + e^3 + e^2 + e} {a-e^2+b-f^2}$$ 
$$d = \frac{a^2b - 2eab -af +b^3 -2b^2f + bf^2 + e^2b + eb + b - f}{a-e^2 + b -f^2} -b + f$$ 
$$g = \frac{a^3 -2ea^2 - a^2 + ab^2 - 2abf + af^2 + e^2a + ea - a -b^2 + bf + e} {a-e^2+b-f^2}$$ 
$$h = \frac{a^2b - 2eab -af + b^3 - 2b^2f + e^2b + eb +b -f} {a-e^2+b-f^2}$$


Examples of where $g\sigma(z) = g\sigma(z+1)$
\\
$$g\sigma(3 + 7i) = 10 + 10i = g\sigma(4 + 7i)$$
$$g\sigma(19 + 25i) = -20 + 60i = g\sigma(20 + 25i)$$
$$g\sigma(19 + 75i) = -100 + 100i = g\sigma(20 + 75i)$$
$$g\sigma(40 + 85i) = -100 + 140i  = g\sigma(41 + 85i)$$
$$g\sigma(90 + 245i) = -320 + 480i  = g\sigma(91 + 245i))$$
\\

\section{Non-Consecutive Gaussian Integers Containg Two Primes}
For $g\sigma(z) = g\sigma(z+k+qi)$ can also be solved using the previous formula. It is easy to show using the previous equations. 
$$n = (a + bi)(c+di), \; \; \; z+k+qi = (e + fi)(g + hi)$$ 
and by $g\sigma$ of primes 
$$ g\sigma(z) = (a + bi + 1)(c + di + 1) = (ac + a + c -bd) + i(ad + d + bc + b)$$
$$ g\sigma(z+k+qi) = (e + fi + 1)(g + hi + 1) = (eg + e + g - fh) + i(eh + fg + h + f)$$ 
and we have the four equations 
$$ac -bd = eg -fh - k$$ 
$$bc + ad = fg + eh - q$$ 
$$ac + a + c -bd = eg + e + g - fh$$ 
$$ad + d + bc + b = eh + fg + h + f$$
Giving the solutions:

$c = \frac{-ea^2 -afq + eak + ak + 2e^2a -eb^2 + bfk + 2ebf + bq + ebq -f^2k -ef^2 -fq-e^2k -ek -e^3}{a^2-2ea+b^2-2bf+f^2+e^2}$

$d = \frac{-a^2b+a^2q+2eab+afk+aq-eaq-b^3+2b^2f+b^2q-bf^2-bfq-ebk-bk-e^2k+fk-ea}{a^2-2ea+b^2-2bf+f^2+e^2} +b -f -q$

$g =\frac{-a^3 + a^2k + 2ea^2-ab^2+2abf-af^2-afq-eak+ak-e^2a+b^2k-bfk+bq+ebq-fq-ek}{a^2-2ea+b^2-2bf+f^2+e^2}$

$h = \frac{-a^2b+a^2q+2eab+afk+aq-eaq-b^3+2b^2f+b^2q-bf^2-bfq-ebk-bk-e^2b+fk-ea}{a^2-2ea+b^2-2bf+f^2+e^2} $
\\
\\
A few solutions for $g\sigma(z) = g\sigma(z+i)$
\\
$$g\sigma(15 + 9i) = 40i = g\sigma(15 + 10i)$$
$$g\sigma(25 + 39i) = -40 + 80i  = g\sigma( 25 + 40i)$$
$$g\sigma(25 + 90i) = -120 + 120i  = g\sigma(25 + 91i)$$
$$g\sigma(65 + 60i) = -40 + 160i = g\sigma(65 + 61i)$$
\\
Lastly, solutions for $g\sigma(z) = g\sigma(z+1+i)$
\\
$$g\sigma(8 + 16i) = 10 + 10i = g\sigma(3 + 12i)$$
$$g\sigma(5 + 5i) = 20i = g\sigma(6 + 6i)$$
$$g\sigma(13 + 4i) = 20i = g\sigma(14 + 5i)$$
$$g\sigma(84 + 5i) = 8 + 96i  = g\sigma(85 + 6i)$$
$$g\sigma(90 + 6i) = -160 + 160i  = g\sigma(91 + 7i))$$
$$g\sigma(138 + 102i) = -400 + 80i = g\sigma(139 + 103i)$$
%$$g\sigma(149 + 53i) = -340 + 180i = g\sigma(150 + 54i)$$

\end{document}  


