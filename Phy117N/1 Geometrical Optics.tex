\documentclass[12pt, oneside]{article}   	% use "amsart" instead of "article" for AMSLaTeX format
\usepackage{geometry}                		% See geometry.pdf to learn the layout options. There are lots.
\geometry{letterpaper}                   		% ... or a4paper or a5paper or ... 
%\geometry{landscape}                		% Activate for for rotated page geometry
%\usepackage[parfill]{parskip}    		% Activate to begin paragraphs with an empty line rather than an indent
\usepackage{graphicx}				% Use pdf, png, jpg, or eps§ with pdflatex; use eps in DVI mode
								% TeX will automatically convert eps --> pdf in pdflatex		
\usepackage{amssymb}
\title{Chapter 1: Geometrical Optics}
\author{Daniel Rice and Thomas Lockwood}
\date{}							% Activate to display a given date or no date

\begin{document}
\maketitle{}
\abstract
The goals in this experiment are to observe the geometrical optics of lights.
Going through many methods we were able to see how light bends after going through different objects.
For example, we were able to see the path of a laser going through a lucite circle.
Instead of going straight through the lucite it bent in a different direction.
From this and other experiments we were able to discover the change of direction of the laser/light by calculating the refraction and reflection.
Prisms just like circles change the directions of light however, with prisms we were able to observe the separation of wavelegnths of colored light.


\subsection*{1.1 The Stenopaic Spectacles}
In this experiment we looked out of a lens with pinholes. 
We were able to observe, after looking into the glasses, a slight dim in brightness and increase of depth perception. 
It was easy to see small print writing. 
I could see the writing on my journal in a zoomed in way. 
The writing was tiny before the glasses, however after the width of the writing's lines was noticeably bigger. 
A hypothesis of why this happens is because the pinholes in the glasses make it easier for our eyes to focus on objects that are as equally small.

\subsection*{1.2 The Laws of Reflection and Refraction}
\includegraphics[scale=.65]{../../Documents/MATLAB/Air2Lucite.jpg} 
\\
\includegraphics[scale=.65]{../../Documents/MATLAB/Air2Lucite2.jpg} 
\\
\includegraphics[scale=.65]{../../Documents/MATLAB/Air2Lucite3.jpg} 
\\
\includegraphics[scale=.65]{../../Documents/MATLAB/Lucite2Air.jpg} 
\\
\includegraphics[scale=.65]{../../Documents/MATLAB/Lucite2Air2.jpg} 
\\
\includegraphics[scale=.65]{../../Documents/MATLAB/Lucite2Air3.jpg} 

\subsection*{1.3.2 Apparent Thickness of a Block of Lucite}
Formula for $S$ and $S`$:
$$ S = P_3 - P_1, \; \; \; S` = P_3 - P_2$$
Our results gives us
$$ P_1 = 1.4\text{mm}\pm.1\text{mm},\;\;\; P_2 = 6.4\text{mm}\pm.1\text{mm},\;\;\; P_3 = 18.9\text{mm}\pm.1\text{mm} $$
$$ S = 18.9 - 1.4 = 17.5\text{mm}, \; \; \; S` = 18.9 - 6.4 = 12.5\text{mm} $$  
Forumula for Index of Refraction:
$$ n = \frac{S}{S`} = \frac{17.5}{12.5} = 1.4 $$
The index of refraction calculated for lucite using the critical angle from the Snell's wheel experiment is 1.49.
The values calculated are only one hundredth apart which suggests that the actual value of the lucite index of refraction is around the value of 1.4.

\subsection*{1.5 The Prism Spectrometer}

\includegraphics[scale=.65]{../../Documents/MATLAB/ColoredLight.jpg} 



\subsection*{1.6.1 Power of a Thin Prism}
Formula for Power:
$$ P = 100\frac{x}{b} $$
Our results gives us
$$ x = 1\text{cm} \pm .1\text{cm}, \; \; \; b = 39.2\text{cm} \pm .1\text{cm}, \; \; \; P = 100(1)/(39.2) = 2.55^{\Delta}$$
Formula for Apex:
$$ \delta = (n-1)\alpha$$
Our results gives us
$$ n = 1.5, \; \; \; \alpha= .57^{\circ}, \; \; \; \delta = (1-1.5).57 = .285^{\circ} $$

\subsection*{1.6.2 The Risley Prism}
Forumula for Lateral Displacement:
$$ x = P\frac{b}{100} $$
$$ P = 6^{\Delta}, \; \; \; b = 44.8\text{cm}, \; \; \; P = 100(44.8)/100 = 2.688\text{cm}$$
Our calculations were $-0.102$cm off from the manual's expected calculated value. 
\subsection*{1.7.1 Internal Reflection in a Bar}
In this experiment we viewed the path of a laser through two different plastics bars.
One was a straight rectangular bar, while the other was curved. 
As the laser traveled through the straight rectangular bar the laser reflected internally several times at approximately the same angle for each reflection before exiting. 
In the curved bar the laser also reflected, but the angle of incidence changed to follow the path of the curve before escaping from the end.
\subsection*{1.7.2 Internal Reflection in a Coil}
In this experiment we viewed the laser as it traveled through a spiral coil.
The coil seemed to illuminate as the laser entered through one end.
The reason could be that the angle of the laster was reflected off of the spiral shape.
It was more difficult to see the laser inside the plastic, however just like the other clear plastic objects it exited through the opposite end.

\subsection*{1.7.3 A Bundle of Fibers}
Unfortunately, we were unable to gather fibers for this portion of the lab. Omit this section.

\subsection*{1.7.4 Coherent Fiber Optics}
In this experiment we used a tool created of jacketed fiber bundles to view a well illuminated word in the lab manual. 
By placing the objective lens of the fiber close to the word, the light was reflected through the fiber optics cable to the eyepiece allowing us to view the image.
The image was still visible despite the curved nature of the tool. 
This occurs due to the fiber jackets having different lower index material to prevent cross-feed of light that would distort the material.

\subsection*{1.7.5 An Illuminator}
We observed the commercial illuminator that had a white-light source.
The light that came from the illuminator reflected off of the fibers through the curve of the object and exited through the end.

\end{document}  