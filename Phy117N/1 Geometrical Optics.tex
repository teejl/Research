\documentclass[12pt, oneside]{article}   	% use "amsart" instead of "article" for AMSLaTeX format
\usepackage{geometry}                		% See geometry.pdf to learn the layout options. There are lots.
\geometry{letterpaper}                   		% ... or a4paper or a5paper or ... 
%\geometry{landscape}                		% Activate for for rotated page geometry
%\usepackage[parfill]{parskip}    		% Activate to begin paragraphs with an empty line rather than an indent
\usepackage{graphicx}				% Use pdf, png, jpg, or eps§ with pdflatex; use eps in DVI mode
								% TeX will automatically convert eps --> pdf in pdflatex		
\usepackage{amssymb}
\title{Chapter 1: Geometrical Optics}
\author{Daniel and Thomas Lockwood}
%\date{}							% Activate to display a given date or no date

\begin{document}
\maketitle{}
\subsection*{1.1 The Stenopaic Spectacles}
In this lab we were instructed to look out of a lens with a bunch of holes. We were able to observe, after looking into the glasses, a slight dim in brightness. It was easy to see small print writing. I could see the writing on my journal in a sort of zoom in. The writing was tiny before the glasses, however after the width was noticeably bigger. 

\subsection*{1.2 The Laws of Reflection and Refraction}


\subsection*{1.3.1 Apparent Depth of a Tank of Water}

\subsection*{1.3.2 Apparent Thickness of a Block of Lucite}

\subsection*{1.5 The Prism Spectrometer}

\subsection*{1.6.1 Power of a Thin Prism}

\subsection*{1.6.2 The Risley Prism}

\subsection*{1.7.1 Internal Reflection in a Bar}

\subsection*{1.7.2 Internal Reflection in a Coil}

\subsection*{1.7.3 A Bundle of Fibers}

\subsection*{1.7.4 Coherent Fiber Optics}

\subsection*{1.7.5 An Illuminator}


\end{document}  